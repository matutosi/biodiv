\documentclass[11pt]{jsarticle}

  % https://joker.hatenablog.com/entry/2012/07/09/153537
\usepackage[top=20truemm,bottom=30truemm,left=25truemm,right=25truemm]{geometry}
  % A4タテで,上2cm,下3cm,左右各2.5cmの余白をとる.
  % 用紙の左上4cm×4cmには講演番号が入るため,ここに文字がかぶらないようにする.

	% 図
\usepackage{graphicx}	% 画像読み込み
\usepackage{mediabb}	% 画像読み込み
\usepackage{wrapfig}	% 回りこみの画像

	% UDフォント
\usepackage{pxchfon}
\setminchofont{UDDIGIKYOKASHON-R.TTC} 
\setgothicfont{UDDIGIKYOKASHON-B.TTC} 

\pagestyle{empty}

\begin{document}
\Large

  % \begin{center}
  % \end{center}
\centerline{植生調査を支援するアプリの開発}

 \\[-12mm]

\normalsize
\rightline{松村俊和(甲南女子大学)}

 \\[-8mm]

[背景]
植生調査をはじめとした生物の現地調査では,長年にわたって紙の調査票への記入が行われている.
紙への記入は簡便,汎用性の高さ,文字情報以外も容易に記入可能といった多くの利点がある.
一方で,階層別の被度の手作業での計算が必要,記録の確認が目視のみ,記入データの再入力の必要性などの欠点がある.

[目的]
紙への記録の欠点を補うことができれば,植生調査の時間短縮や効率化が可能である.
そこで植生調査を支援するアプリを開発した.
開発の際には,紙への記入の欠点を補うとともに,以下の点を特に考慮した.
\begin{itemize}
\item 汎用性の高いものとする(植生調査以外でも利用可能).
\item 使用者が項目の設定および保存・復元が可能.
\item OSによらず利用でき,シェアの高いブラウザで動作.
\item オフラインでも動作およびデータの保存が可能.
\item 外部のライブラリやフレームワークを使用しない.
\item ソースコードを公開して,使用者が機能を追加・修正可能.
\end{itemize}

[使用言語等]
HTML, JavaScript, CSS (htmlファイルにJavaScripとCSSを格納済み)

[動作確認環境]
GoogleChrome

[その他]
データは電子メールで添付ファイルとして送信する.
ファイル形式はJSON形式を基本としたテキストファイルとRで取り込み可能なものとする.
また,csvやtsvとしても保存・送信できるようにする.


[インストール・起動]
以下のファイルディレクトリのファイルを各自の端末にダウンロードする.

https://github.com/matutosi/biodiv/tree/main/www/biss.html

biss.html を起動する.


[使用方法]
Settingsタブで,

Inputsタブで,地点情報や観察情報を入力する.
観察情報では以下が可能.
・テキストデータの検索,並び替え,列の表示・非表示,追加.
・追加後の日時・GPSデータの更新

[出力されるファイルの形式]
設定ファイルおよび入力データともに,列名・データ形式・選択リスト・データの4項目からなるテキストファイルである.
各項目はJSON形式であり,";"(セミコロン)で区切られている.
Rではテキストファイルを読み込み,";"で区切れば,jsonlite等のパッケージを使ってデータフレームへの変換ができる.
  % 筆者作成の関数(ecan::read_biss())も適宜,ご利用ください.

[]
https://github.com/matutosi/ecan

ダミーのテキスト1
ダミーのテキスト2
ダミーのテキスト3
ダミーのテキスト4
ダミーのテキスト5
ダミーのテキスト6
ダミーのテキスト7
ダミーのテキスト8
ダミーのテキスト9
ダミーのテキスト10
ダミーのテキスト11
ダミーのテキスト12
ダミーのテキスト13
ダミーのテキスト14
ダミーのテキスト15
ダミーのテキスト16
ダミーのテキスト17
ダミーのテキスト18
ダミーのテキスト19
ダミーのテキスト20
ダミーのテキスト21
ダミーのテキスト22
ダミーのテキスト23
ダミーのテキスト24
ダミーのテキスト25
ダミーのテキスト26
ダミーのテキスト27
ダミーのテキスト28
ダミーのテキスト29
ダミーのテキスト30
  % ダミーのテキスト31
  % ダミーのテキスト32
  % ダミーのテキスト33
  % ダミーのテキスト34
  % ダミーのテキスト35
  % ダミーのテキスト36
  % ダミーのテキスト37
  % ダミーのテキスト38
  % ダミーのテキスト39
\end{document}
